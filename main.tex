\documentclass{article}
 \usepackage[spanish]{babel}
 \usepackage{amsmath,amssymb}
\usepackage[amssymb]{SIunits}
%\usepackage{enumerate}
% \usepackage{lipsum}
% \usepackage{natbib}
\usepackage{graphicx}
\usepackage{analysis_orax}
\usepackage{subcaption}
\usepackage{hyperref}
\usepackage{listings}
\usepackage{marvosym}
\usepackage{mdframed}

\title{Cuantización de campos masivos}

\author{Juan Diego Garro\footnote{\href{mailto:diego.garro@udea.edu.co}{diego.garro@udea.edu.co}}\\
\textit{\small  Instituto de Física, Universidad de Antioquia}
}
\date{\small Diciembre 10, 2020}
\begin{document}
\lstset{language=[LaTeX]TeX
%,showspaces=false
%{\small Hola} {\footnotesize Hola} {\scriptsize Hola} {\tiny Hola}
,basicstyle=\ttfamily
,showstringspaces=false
,keywordstyle=\color{blue}
,commentstyle=\normalfont\ttfamily\color{red}
,stringstyle=\color{Green}
,inputencoding=utf8
,extendedchars=false
%,framexleftmargin=2mm 
,frame=shadowbox 
%,rulexsepcolor=\color{blue}
}

\maketitle

\begin{abstract}
En la teoría cuántica de campos un campo clásico pasa a ser un operador de campo cuántico que es tratado bajo el marco de Heisenberg, nuestras variables pasan a ser los campos fundamentales, que permean todo el universo, asociados a las partículas subatómicas fundamentales y para estos campos y su respectivo campo de momentum conjugado se imponen las relaciones de conmutación canónicas. A este proceso también se le denomina segunda cuantización y da lugar, entre otras cosas, a la cuantización de estados de múltiples partículas, a diferencia de la primera cuantización donde se cuantiza la energía y los estados de una sola partícula.
\end{abstract}

%\section{Aclaración}
%Una vez recibido este documento por favor cree una copia del mismo para futura 
%referencia y trabaje directamente sobre éste, borrando todos los contenidos del mismo. 
%Esto con el fin de poder activar el sistema automático de revisión de Overleaf 
%que requiere una cuenta especial desde la cual el presente documento fue creado. 
\section{Introducción}
Los campos cuánticos permean todo el universo y son mas fundamentales que las partículas subatómicas, pues estas partículas se consideran excitaciones o superposición de oscilaciones de sus respectivos campos \cite{Restrepo:2020}. Al ser campos oscilatorios, los campos de las partículas fundamentales pueden ser cuantizados al igual que ocurre por ejemplo en el problema del oscilador armónico en mecánica cuántica.

\subsection{Definición del Campo}
Es una cantidad física definida en todo el espacio, se puede considerar como
una colección de infinitos osciladores distribuidos uniformemente, las variables del lagrangiano $\phi$ pueden representar cualquier cantidad (u objeto) física oscilante, cada punto oscila independientemente en el espacio-tiempo o bien, una oscilación que se propaga ejerciendo fuerza sobre las cantidades físicas a su alrededor.

\section{Algunas formas de cuantizar el campo}
La cuantización de campos se puede lograr de varias maneras, como a través de integrales de camino o usando el programa de cuantización de Batalin Vilkovisky \cite{Kaku:1993}. La forma que se desarrollará en este documento es análoga a como se realiza la cuantización en mecánica cuántica. 

La cuantización de campos se puede lograr, resumidamente, de la siguiente manera:
\begin{itemize}
\item interpretando los campos clásicos como operadores de campos cuánticos en el marco de Heisenberg (los operadores evolucionan temporalmente) y 
\item imponiendo relaciones de conmutación en tiempos iguales para el campo y su campo de momentum conjugado.
\end{itemize}
Sobre notación: $x$ denota el espacio-tiempo de cuatro dimensiones y $\mathbf{x}$ las tres dimensiones espaciales.
El campo puede ser visto como una colección de osciladores discretos, la cuantización se puede realizar en un esquema de mecánica cuántica, tal que los conmutadores del campo se pueden obtener reemplazando $q_r, p_s$ por $\phi(t,\mathbf{x}_r) \Delta V, \pi(t,\mathbf{x}_s)$ en las relaciones de conmutación canónicas para las posiciones y el momentum (teniendo en cuenta que tomamos el limite $\Delta V \rightarrow 0 $)

\begin{equation}
  [\phi(t,\mathbf{x}_r),\pi(t,\mathbf{x}_s)] = \frac{i \delta_{rs}}{\Delta V} \longrightarrow  [\phi(t,\mathbf{x}),\pi(t,\mathbf{y})] = i\delta^3(\mathbf{x}-\mathbf{y}).
\end{equation}

\section{Cuantización de campos escalares complejos}
Para un campo escalar complejo de spin $0$, el campo complejo se pone en términos de 2 campos reales implicando que tiene dos grados de libertad internos $\phi =\varphi_1 + i\varphi_2$, pero podemos tomar como campos independientes $\phi$ y  $\phi^\dagger$. El lagrangiano esta dado por $\mathscr{L_{KG,c}}=\partial_\mu \phi^\dagger \partial^\mu \phi - m^2 \phi^\dagger \phi$. El siguiente proceso especifico de cuantizacion de este campo es tomado de \cite{Nagashima:2010}.
Las cantidades físicas conservadas que son los operadores de energía (Hamiltoniano), momentum y carga (dicha carga surge de la simetría del grupo U(1)) son
%
\begin{equation}
  H= \sum p_r\dot{q}_r - L \rightarrow H=\pi(x) \dot{\phi} (x) - L = \iiint  d^3x \  [(\frac{\partial \phi^\dagger}{\partial t})(\frac{\partial \phi}{\partial t})+(\nabla \phi^\dagger \cdot \nabla \phi) + m^2 \phi^\dagger \phi]\,,
\end{equation}
\begin{equation}
  \mathbf{P}=-\iiint  d^3x \  [\frac{\partial \phi^\dagger}{\partial t} \nabla \phi +
  \nabla \phi^\dagger \frac{\partial \phi}{\partial t}]\,,
\end{equation}
\begin{equation}
  Q= i q \iiint  d^3x \  [ \phi^\dagger \frac{\partial \phi }{\partial t} - \frac{\partial \phi^\dagger }{\partial t}\phi].
\end{equation}

Ya que la ecuación de Klein-Gordon es una ecuación de onda con primeras derivadas, podemos expandir su solución  en la base de ondas planas con numero de onda $\overrightarrow{k}=\mathbf{k}$ 
%
\begin{equation}
  \phi(x) = \sum_{\mathbf{k}} q_{\mathbf{k}}(t) e^{i\mathbf{k} \cdot \mathbf{x}}.
\end{equation}

Al aplicarle la ecuación de Klein-Gordon a la expansión 
%
\begin{equation}
  [\partial_\mu \partial^\mu + m^2] \phi = \sum_{\mathbf{k}} (\ddot{q_{\mathbf{k}}} + w^2q_{\mathbf{k}}) e^{i\mathbf{k} \cdot \mathbf{x}} = 0\,, \ \ \ w=\sqrt{\mathbf{k}^2+m^2}\,,
\end{equation}
%
vemos que el campo que satisface dicha ecuación es una suma de un numero infinito de osciladores con diferente numero de onda $\mathbf{k}$ (están diferentes modos de oscilación, diferentes frecuencias implica que son independientes unos de otros) por lo que los coeficientes son ondas con frecuencia $w$. Separando $q_\mathbf{k}$ en partes con frecuencia positiva y partes con frecuencia negativa
%
\begin{equation}
  \phi(x) = \sum_{\mathbf{k}} \frac{1}{\sqrt{2wV}} (a_{\mathbf{k}}e^{-iwt}+c_{\mathbf{k}}e^{iwt}) e^{i\mathbf{k} \cdot \mathbf{x}}\,,
\end{equation}
donde V es el volumen de normalización de la onda planas.

Si introducimos esta expansión en las expresiones para los operadores de Hamiltoniano, momentum y carga (Eqs. (2), (3) y (4) respectivamente) y usamos la relación de ortogonalidad de la base discreta entre $\mathbf{k}$ y $\mathbf{k'}$
%
\begin{equation}
  \frac{1}{V}\iiint  d^3x \   e^{i(\mathbf{k}-\mathbf{k'}) \cdot \mathbf{x}}=\delta_{\mathbf{kk'}}.
\end{equation}

Al multiplicar sumatorias, estas se ponen con diferentes índices $\mathbf{k}$ y $\mathbf{k'}$ y una de estas sumatorias desaparece al ser integradas en el espacio con la relación de ortogonalidad anterior. Se llega a 
%
\begin{equation}
  H = \sum_{\mathbf{k}} w[a_{\mathbf{k}}^\dagger a_{\mathbf{k}} + c_{\mathbf{k}}^\dagger c_{\mathbf{k}} ]\,,
\end{equation}

\begin{equation}
  \mathbf{P} = \sum_{{\mathbf{k}}} \mathbf{k}[a_{\mathbf{k}}^\dagger a_{\mathbf{k}} - c_{\mathbf{k}}^\dagger c_{\mathbf{k}} ]\,,
\end{equation}

\begin{equation}
  Q = \sum_{{\mathbf{k}}} q[a_{\mathbf{k}}^\dagger a_{\mathbf{k}} - c_{\mathbf{k}}^\dagger c_{\mathbf{k}} ].
\end{equation}

Como el campo satisface  relaciones de conmutación canónicas en tiempos iguales los modos de Furier (los coeficientes operador-valuados de la expansión de Fourier) también deben satisfacer relaciones de conmutación:
%
\begin{equation}
  [a_{\mathbf{k}},a_{{\mathbf{k}}'}^\dagger] = \delta_{\mathbf{kk'}}\,, \ \  [c_{\mathbf{k}},c_{{\mathbf{k}}'}^\dagger] = \delta_{\mathbf{kk'}}\,, 
\end{equation}
%
todos los demas conmutadores entre $a_{\mathbf{k}},a_{\mathbf{k}}^\dagger,c_{\mathbf{k}},c_{\mathbf{k}}^\dagger$ son cero. 

Aquí $(a_{\mathbf{k}}^\dagger, a_{\mathbf{k}})$ se pueden interpretar como operadores de creación y aniquilación de una partícula con energía-momentum ($E=\hbar w \,, \ \mathbf{P}=\hbar \mathbf{k} $). Pero para la parte de frecuencia negativa,   $(c_{\mathbf{k}}^\dagger, c_{\mathbf{k}})$ son operadores de creación y aniquilación de una partícula con energía negativa, lo cual no es permitido ya que no debe haber partículas con energía negativa. Debemos examinar los efectos de aplicar estos operadores en vectores de estado. Definimos el operador numero $N$ como
%
\begin{equation}
  N= \sum_{\mathbf{k}}(a_{\mathbf{k}}^\dagger a_{\mathbf{k}} + c_{\mathbf{k}}^\dagger c_{\mathbf{k}} )\,,
\end{equation}
en analogía con el oscilador armónico (para el cual $N=a^\dagger a$) el operador numero es la sumatoria para los diferentes momentos $\mathbf{k}$ del producto anterior y del producto para la parte de las frecuencias negativas.

Usando las relaciones de conmutación Eq.(12) obtenemos las relaciones de conmutación entre N y $a_{\mathbf{k}},a_{\mathbf{k}}^\dagger,c_{\mathbf{k}},c_{\mathbf{k}}^\dagger$ 
%
\begin{equation}
  [N,a_{{\mathbf{k}}'}^\dagger] = a_{{\mathbf{k}}'}^\dagger\,, \ \  [N,a_{{\mathbf{k}}'}] = -a_{{\mathbf{k}}'}\,, 
\end{equation}
%
\begin{equation}
  [N,c_{{\mathbf{k}}'}^\dagger] = c_{{\mathbf{k}}'}^\dagger\,, \ \  [N,c_{{\mathbf{k}}'}] = -c_{{\mathbf{k}}'}\,,
\end{equation}
%
que muestran que $a_{{\mathbf{k}}'}^\dagger, c_{{\mathbf{k}}'}^\dagger$ se pueden interpretar como operadores de creación y $a_{{\mathbf{k}}'}, c_{{\mathbf{k}}'}$ operadores de aniquilación de partículas con momentum $\mathbf{k}$.

En el cuadro de Heisenberg, los operadores evolucionan en el tiempo, si tomamos la evolución temporal del campo $\phi$ actuando sobre un autoestado de energía $|E\rangle$ de $H |E\rangle = E |E\rangle$
%
\begin{equation}
 \dot{\phi} |E\rangle = i[H,\phi] |E\rangle\,,
\end{equation}
%
comparando los coeficientes de $e^{-iwt}$ y $e^{iwt}$ y usando las relaciones de conmutación entre N y los coeficientes Eqs. (14 y 15) tenemos que
%
\begin{equation}
 -iwa_{\mathbf{k}} |E\rangle = i[H,a_{\mathbf{k}}] |E\rangle  =i(H-E)a_{\mathbf{k}} |E\rangle\,,
\end{equation}
%
\begin{equation}
 iwc_{\mathbf{k}} |E\rangle = i[H,c_{\mathbf{k}}] |E\rangle  =i(H-E)c_{\mathbf{k}} |E\rangle\,,
\end{equation}
%
de donde obtenemos
%
\begin{equation}
 Ha_{\mathbf{k}} |E\rangle =(E-w)a_{\mathbf{k}} |E\rangle\,,
\end{equation}

\begin{equation}
 Hc_{\mathbf{k}} |E\rangle =(E+w)c_{\mathbf{k}} |E\rangle\,.
\end{equation}

Vemos que el operador $a_{\mathbf{k}}$ (coeficiente de ondas con frecuencia positiva) reduce el numero de partículas en $1$ y también reduce la energía del estado $|E\rangle$ en $w$, por lo que efectivamente es un operador de aniquilación. Pero  $c_{\mathbf{k}}$ (coeficiente de ondas con frecuencia negativa) incrementa la energía del estado en $w$, entonces este debe crear una partícula y debe ser considerado como un operador de creación. Por esto redefinimos
$c_{\mathbf{-k}} \rightarrow b_{\mathbf{k}}^\dagger\,, \  c_{\mathbf{-k}}^\dagger \rightarrow b_{\mathbf{k}}$ con sus respectivas relaciones de conmutación 
%
\begin{equation}
  [b_{\mathbf{k}},b_{{\mathbf{k}}'}^\dagger] = \delta_{\mathbf{kk'}}\,, \ \  [b_{\mathbf{k}},b_{{\mathbf{k}}'}] = [b_{\mathbf{k}}^\dagger,b_{{\mathbf{k}}'}^\dagger] = 0\,,
\end{equation}
y los conmutadores de $b_{\mathbf{k}}$ con $a_{\mathbf{k}}$ también son cero. La razón para tomar $\mathbf{-k}$ es hacer el segundo termino en el momentum $\mathbf{P}$ positivo (lo cual se verifica al reemplazar la redefinición del campo Eq. (22), donde se cambia $c_\mathbf{k}$ por $c_\mathbf{-k}$ = $b_\mathbf{k}^\dagger$, en la ecuación de momentum Eq. (3), el resultado es que en lugar de $-c_\mathbf{k}^\dagger c_\mathbf{k}$ queda $b_\mathbf{k}^\dagger b_\mathbf{k}$ ) de manera que la partícula tenga tanto momentum positivo como energía positiva. Con esto se redefine el campo como

\begin{equation}
  \phi(x) = \sum_{\mathbf{k}} \frac{1}{\sqrt{2wV}} (a_{\mathbf{k}}e^{-iwt}+c_{\mathbf{k}}e^{iwt}) e^{i\mathbf{k} \cdot \mathbf{x}} = \sum_{\mathbf{k}} \frac{1}{\sqrt{2wV}} (a_{\mathbf{k}}e^{-iwt+i\mathbf{k} \cdot \mathbf{x}}+b_{\mathbf{k}}^\dagger e^{iwt-i\mathbf{k} \cdot \mathbf{x}}).
\end{equation}

También los operadores Hamiltoniano, momentum y carga, en términos de los nuevos operadores de creación y aniquilación, son reescritos como
%
\begin{equation}
  H = \sum_{\mathbf{k}} w[a_{\mathbf{k}}^\dagger a_{\mathbf{k}} + b_{\mathbf{k}}^\dagger b_{\mathbf{k}} + 1]\,,
\end{equation}

\begin{equation}
  \mathbf{P} = \sum_{{\mathbf{k}}} \mathbf{k}[a_{\mathbf{k}}^\dagger a_{\mathbf{k}} + b_{\mathbf{k}}^\dagger b_{\mathbf{k}} + 1]\,,
\end{equation}

\begin{equation}
  Q = \sum_{{\mathbf{k}}} q[a_{\mathbf{k}}^\dagger a_{\mathbf{k}} - b_{\mathbf{k}}^\dagger b_{\mathbf{k}} - 1]\,,
\end{equation}

Estas expresiones nos dicen que $b$ es un operador de aniquilación  de una partícula que tiene energía-momentum $(w,\mathbf{k})$ pero con carga opuesta, que llamaremos antipartícula, por lo que este nuevo estado de la materia también tiene energía positiva. Feynman interpretó estas partículas con energía negativa moviéndose hacia atrás en el tiempo como antipartículas con energía positiva moviéndose hacia adelante en el tiempo \cite{Kaku:1993}.
Con estas cuatro últimas ecuaciones y las relaciones de conmutación de los coeficientes (operadores creación y aniquilación) se puede probar directamente

\begin{equation}
  [\phi(x),\pi(y)]_{t_x=t_y} = i \delta^3(\mathbf{x}-\mathbf{y})\,,
\end{equation}

\begin{equation}
  [\phi(x),\phi(y)]_{t_x=t_y} =
  [\pi(x),\pi(y)]_{t_x=t_y} = 0.
\end{equation}

Con

\begin{equation}
  \pi(x)=\frac{\partial\mathscr{L}}{\partial( \dot{\phi}(x))}=\dot{\phi}^\dagger (x).
\end{equation}

\section{Discusión}
La segunda cuantización se puede interpretar como la cuantización de los diferentes modos de vibración con momentum $k$ en los que se puede descomponer el campo, esto lleva a estados cuánticos discretos para el campo, donde estos estados no representan a una partícula, si no un numero arbitrario de partículas con su respectivo momentum, es decir un solo estado del campo esta asociado a varias partículas, es un estado antipartículas y son estos estados los que son discretizados en la cuantización \cite{Kaku:1993}.

Otra interpretación de esta teoría es que hay infinitos escenarios simultáneos del campo y de sus excitaciones interactuando  y es la superposición de todos estos (cada escenario con su respectivo "pesos")  la que se manifiesta físicamente \cite{Roussel:2020}.

%\section{Cuantización del campo de spinores de Dirac}
\section{Discusión sobre spinores de Dirac}
Al desarrollar la cuantización del campo de Dirac, se propone una interpretación correspondiente a la creación de un electron y un antielectrón (positrón). El problema inicial es que todos los electrones de la naturaleza tienen una probabilidad finita de decaer al estado mínimo de energía posible, esto pasaría eventualmente pero no es consistente con la realidad, por lo que se propone que el vacío es en realidad un mar estados de mínima energía llenos y por esta razón los electrones no pueden decaer hasta esta energía mínima, pues todos los estados están ocupados por electrones y solo un electrón puede ocupar un estado. Cuando un fotón choca contra este mar, logra desprender un electrón dejando un $"hueco"$ en su lugar, el cual sería el positrón \cite{Kaku:1993}, esto es una forma de visualizar una de las reglas de Feynman, el proceso de creación en que un fotón se convierte en un electrón y un positrón \cite{Restrepo:2020} que luego podrían aniquilarse entre ellos.

\section{Referencias}

\bibliographystyle{apsrev4-1long}
\bibliography{susy}


\end{document}
